\documentclass[11pt,a4paper]{moderncv}
\usepackage{libertine}
% sudo apt-get install fonts-font-awesome

\usepackage{fontspec} % что-то про шрифты?
\usepackage{polyglossia} % русификация xelatex

\setmainlanguage{russian}
\setotherlanguages{english}

% download "Linux Libertine" fonts:
% http://www.linuxlibertine.org/index.php?id=91&L=1
%\setmainfont{Linux Libertine O} % or Helvetica, Arial, Cambria
% why do we need \newfontfamily:
% http://tex.stackexchange.com/questions/91507/
%\newfontfamily{\cyrillicfonttt}{Linux Libertine O}


\usepackage{geometry}% http://ctan.org/pkg/geometry
\geometry{left=1cm, right=1cm}

\moderncvtheme[blue]{casual}

\firstname{Борис}
\familyname{Демешев}
% \title{Title}              % Your title (optional)
%\address{Новопесчаная ул. 19, кв. 155}{125252, Москва, Россия} % Your current address
\mobile{+7 903 287 34 22}
%\phone {number}    % Your phone number
\email{boris.demeshev@gmail.com}          % Your email address

%\quote{\tiny  We have not succeeded in answering all our problems. The answers we have found only serve us to raise a whole set of new questions. In some ways we are as confused as ever. But we believe we are confused at a higher level and about more important things}

\homepage{github.com/bdemeshev/}    % Your website
%\extrainfo {information} % Possible extra information e.g. website


\usepackage[%
 backend=biber,% or bibtex8
 style=authoryear,%
 dashed=false, % заменять ли одинаковых авторов на тире
 sorting=ydnt,% sorted by year, descending
]{biblatex}


\renewbibmacro*{date}{}
\renewbibmacro*{date+extrayear}{}
\renewbibmacro*{issue+date}{}
\newcommand*{\bibyear}{}

\defbibenvironment{bibliography}
  {\list
     {\iffieldequals{year}{\bibyear}
        {}
        {\printfield{year}%
         \savefield{year}{\bibyear}}}
     {\setlength{\topsep}{0pt}% layout parameters based on moderncvstyleclassic.sty
      \setlength{\labelwidth}{\hintscolumnwidth}%
      \setlength{\labelsep}{\separatorcolumnwidth}%
      \setlength{\itemsep}{\bibitemsep}%
      \leftmargin\labelwidth%
      \advance\leftmargin\labelsep}%
      \sloppy\clubpenalty4000\widowpenalty4000}
  {\endlist}
  {\item}




\addbibresource{demeshev_rus.bib}



\begin{document}

\makecvtitle

\section{Опыт работы}
\cvline{09.2001 — \ldots}{\small старший преподаватель, НИУ-ВШЭ, эконометрика, теория игр, теория вероятностей и математической статистики, анализ временных рядов, стохастический анализ}
\cvline{09.2003 — \ldots}{\small старший преподаватель, Международный институт экономики и финансов, линейная алгебра, математика для экономистов, стохастический анализ}
\cvline{02.2025 — \ldots}{\small старший преподаватель, МГУ, теория вероятностей}
\cvline{09.2023 — 07.2024}{\small старший преподаватель, Университет МГУ-ППИ в Шеньжене, линейная алгебра, теория вероятностей, математическая статистика}
\cvline{02.2015 — 07.2022}{\small старший преподаватель, РАНХиГС, случайные процессы, эконометрика}

\cvline{09.2019}{\small семинары по математической статистике для ЦБ РФ}
\cvline{09.2018}{\small семинары по машинному обучению для ООО «Интернет решения»}
\cvline{01.2016 — 05.2016}{\small преподаватель, «Эконометрика» на платформе \url{https://openedu.ru/}}
\cvline{06.2015}{\small семинары по R и эконометрике для IMS Health}
\cvline{03.2015 — 06.2015}{\small преподаватель, «Эконометрика» на платформе \url{www.coursera.org}}
\cvline{09.2009 — 08.2011}{\small преподаватель, Католический университет Луван-ля-Нёв, теория вероятностей и математическая статистика, анализ качественных данных, лог-линейные модели}
\cvline{11.2008}{\small анализ макроэкономических факторов влияющих на продажи электроники в России для Самсунг-Россия}
\cvline{06.2007}{\small семинары по математической статистике для ОАО МТС-Россия}
\cvline{11.2006}{\small семинары по анализу временных рядов для ЦБ РФ}
\cvline{05.2006}{\small семинары по математической статистике для ЗАО Транстелеком}


\section{Образование}
\cvline{09.2014}{повышение квалификации, Эконометрика в R}
\cvline{06.2014}{повышение квалификации, Пространственная эконометрика}
\cvline{2009 — 2011}{Курсы по статистике Католического университета Луван-ля-Нёв}
\cvline{2004}{Летняя школа Лондонской Школы Экономики, Опционы, фьючерсы и прочие финансовые инструменты,  А+}
\cvline{2004}{магистр экономики Роттердамской Школы Экономики}
\cvline{2003}{ Летняя школа Лондонской Школы Экономики, Продвинутая эконометрика, А-}
\cvline{2001 — 2003}{магистр экономики, НИУ-ВШЭ, математические методы анализа экономики}
\cvline{1997 — 2001}{бакалавр экономики, НИУ-ВШЭ, диплом с отличием}
\cvline{2000 — 2001}{стажировка, Сорбонна (Париж-1)}
%\cvline{2001 — 2004}{вольнослушатель Независимого Московского Университета}
\cvline{1986 — 1997}{школа \textnumero 1251 с углубленным изучением французского языка,  серебряная медаль}

\newpage

\section{Навыки}

\cvline{Языки:}{\small английский (свободно), французский (свободно), русский (родной) }
\cvline{Компьютер:}{\small R, python, \LaTeX, markdown, git, gretl, eviews, stata, office}



\section{Разное}
\cvline{Хобби}{\small Любительский интерес к генетике, астрономии и квантовой физике}
\cvline{Девиз}{\small  Нам не удалось решить все наши задачи. Ответы, что мы находим, лишь
ставят перед нами новые вопросы. В каком-то смысле мы также ничего не знаем как и раньше.
Но мы верим, что наше незнание стало глубже, а не знаем мы всё более и более важные вещи.}

\nocite{*}
\printbibliography[title={Публикации}]

\end{document}
